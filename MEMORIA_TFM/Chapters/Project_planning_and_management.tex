\chapter{Project planning and Economic study}\label{ch:ProjectPlanningMannagement}

	\section{Project planning and management}
		The project was developed over a period of approximately three months. Although consideration of the research problem and hypothesis began about one month earlier, the effective working period was limited to three months. Within this timeframe, the theoretical and practical components of the project had to be completed.
		
		Due to the limited time, the work was divided to proceed in parallel. On the one hand, the written component was developed progressively throughout the three-month period. As the project evolved and its overall approach became clearer, the report was reviewed. It was rewritten multiple times. This was done to ensure that it accurately reflected the current state of the work.
		
		On the other hand, the practical component was structured into distinct phases. Since the project lacked direct references or established precedents, organizing the work into phases was necessary to define a clear methodology, manage uncertainty, and ensure that all objectives were completed within the established timeframe.
		
		
		\begin{itemize}
			\item{\textbf{Text research \& concept definition (Weeks 1–2):}} This phase represented the initial stage of the project. Its primary objective was to develop a plausible concept based on a clearly identified problem. Since an analysis of the visual landscape had already been conducted, the remaining task was to formulate ideas that directly addressed the issues that had been identified.
			
			The outcome of this phase is presented in the \ref{sec:Hypostesis} section, where the problem is formally defined and the development of a functional tool for visual creation is proposed.
			
			\item{\textbf{Data structuring \& how to create the tool (Weeks 3-8): }} This phase proved to be the most problematic and challenging of the entire project. The objective was to create a tool that integrated artificial intelligence into TouchDesigner; however, at the outset, there was no clear methodological approach for achieving this integration. As explained in training the LLM section \ref{sec:TrainingLLM}, several strategies were explored to determine how LLM-based models could be trained and effectively incorporated into the TouchDesigner environment.
			
			Over the course of several weeks, different possibilities were tested and evaluated. Ultimately, it was determined that the most suitable solution was to use LOPs, training the agents with HTML files as their primary knowledge source. Reaching this decision required extensive experimentation and analysis, and the process took approximately two weeks before a clear direction could be established.
			
			Once the roadmap had been defined, a significant amount of time was dedicated to learning the fundamentals of LOPs. This included understanding how to install the system, how agents function, how each operator works, and how to design an effective workflow. This learning phase involved extensive testing and experimentation and lasted an additional couple of weeks before the actual construction of the system could begin.
			
			\item{\textbf{Creation of the tool (Weeks 9-12): }} This phase represented the final stage of the project and was likely the one that would have required the most time under different circumstances. Due to the project deadline, only four weeks were available to begin configuring the TouchDesigner network in combination with the LLM models. Given the complexity and broad scope of the proposed tool, this phase could have extended significantly longer, as the system offers substantial potential for further development and refinement.
			
			Within the limited timeframe, the result was a preliminary version of the tool that was relatively simple and not yet fully automated. Nevertheless, during these four weeks, it was possible to successfully generate visuals centered around a shader-based aesthetic, drawing on color palettes derived from different artists. Although the implementation remained at an early stage, this phase demonstrated the feasibility of the approach and laid the groundwork for future improvements and expansion.
			
			 
		\end{itemize}
		
	\section{Economic study of the work perfomed}
		The project lasted approximately two and a half months and was completed by two people. During this period, they designed and developed a software tool that enables the integration of Large Language Models (LLMs) into TouchDesigner, allowing for their use in creative and interactive environments.
		
		Regarding direct economic costs, the project was completed on a very limited budget, primarily consisting of software subscription fees.
		
		\begin{itemize}
			\item {Cursor: 15€ per month}
			\item{DotSimulate: 9€ per month.}
			
		\end{itemize}
		
	
		
		
		Given the project's total duration of two and a half months, the overall subscription costs were as follows in figure x:
		
		\begin{figure}[H]
			\centering
			\includegraphics[width=1\linewidth]{gfx/graficoDinero.png}
			\caption{Graphic of the amount of money spent on the project.}
			\label{fig:GraficoDinero}
		\end{figure}
		
		
		Additionally, two Gemini API keys were used during the development process. These were obtained free of charge, so they did not generate any additional costs.
		
		No expenses related to hardware, infrastructure, or additional software licenses were incurred. The primary contributions to the project were the time, expertise, and technical knowledge provided by the two developers. These contributions were not monetized in this economic analysis because the project was conducted in an experimental, research-oriented context.
		
		In conclusion, this project demonstrates that a functional tool integrating TouchDesigner with LLM-based systems can be developed at a very low cost by relying primarily on accessible software solutions and free API services.