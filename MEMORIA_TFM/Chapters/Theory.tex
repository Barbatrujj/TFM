\chapter{Theory}\label{ch:Theory}
\section {Visual history}

	The world of screen visuals is a rapidly evolving field that brings together art, technology, and design. Far from being limited to the simple reproduction of images, this discipline encompasses a wide range of techniques, styles, and audiovisual tools. It includes everything from digital manipulation and real-time rendering to generative art and immersive installations. Today, visuals play a central role in technological art, contributing not only to aesthetic expression but also to musical and interactive experiences.
	
	The origins of this practice can be traced back to the 1960s with the introduction of Sony’s Portapak, the first portable analog video recorder, shown in figure \ref{fig: Portapack}. This device made it possible for creators to experiment with video outside of traditional television studios, opening the door to new forms of artistic expression. Among the first to explore its possibilities was Nam June Paik, who used magnets and electromagnetic filters to distort electronic images and create innovative visual effects\cite{rush2007video,hanhardt1986paik}. At the same time, artists such as Steina and Woody Vasulka focused on developing analog synthesizers capable of modulating waves and altering electronic signals in real time \cite{vasulka1992dialogue}. Although these early explorations often intersected with scientific experimentation, they laid the foundation for what would eventually become a recognized artistic field.
	
	\begin{figure} [H]
		\centering
		\includegraphics[width=0.5\linewidth]{gfx/Portapack.jpg}
		 \caption{Sony Portapak camera (Maison de la Vidéo \& du Cinéma: \cite{portapak_mvc})}
		\label{fig: Portapack}
	\end{figure}
	
	
	
	
	During the 1980s and 1990s, with the advent of digital video, new techniques for editing, production, and temporal manipulation emerged. At this point, artists such as Bill Viola began to explore new techniques using extreme video slow motion and careful compositions \cite{viola1995reasons}. Pipilotti Rist experimented with modifying images by altering saturation and distorting them to create expansive projections \cite{tribe2006newmedia}.
	At the same time, in the field of electronic music, VJing was invented, a practice that consisted of mixing images in real time while music was playing. This concept quickly grew with the development of software such as Modul8, VDMX, and Resolume \cite{spinrad2005vjbook}.
	
	\vspace{0.5cm}
	Starting in the early 2000s, the availability of powerful computers, advanced graphics cards, and high-brightness projectors profoundly transformed visual creation processes. These tools, which were once static, have evolved into dynamic systems capable of generating images in real time, responding to external data, and integrating with sensors or interactive devices. Rather than approaching visual creation as a fixed, pre-defined product, it became possible to use environments such as TouchDesigner, Max/MSP/Jitter, and Pure Data to create a visual experience as a modulable data flow \cite{cycling2020max}.
	
	Concurrently, languages such as Processing and p5.js enabled the growth of generative art, empowering algorithms, mathematical frameworks, and computational logic to dictate the structure and behavior of images \cite{pearson2011generative}. Consequently, the role of the visual artist underwent an evolution, becoming a hybrid profile that integrates design competencies with programming acumen and technical experimentation. Visual creators shifted from operating tools to becoming system architects who can manage complex information flows to produce flexible, interactive audiovisual experiences.
	\vspace{0.5cm}
	
	Today, screens have expanded far beyond the traditional rectangular device, with new options including curved screens, foldable screens, and screens with various sizes and shapes. These elements can now manifest in various forms, including architectural surfaces, immersive environments, interactive installations, and high-resolution urban displays. Sensors, depth cameras, body-tracking devices, and real-time analysis systems are integrated to allow artworks to react to the presence and actions of the audience, creating multisensory experiences in which the image behaves like a living environment. As Paul \cite{paul2015digital} notes, contemporary digital art operates in a hybrid space where physical materiality and computational logic converge, thereby transforming the relationship between viewer and artwork.
	
	This landscape is defined by its technical diversity and complexity. A variety of tools and methods from different fields are used to create screen-based visuals. These fields include composition, animation, programming, interaction design, data visualization, digital scenography, and algorithmic systems. Therefore, an interdisciplinary territory is worked in by the contemporary artist, and mastery of both aesthetic strategies and technological capabilities is required. The screen is no longer merely a display surface but an expanded field where human creativity, electronic processes, and computational structures intersect. The way it keeps changing reflects the big impact of new technology on modern art and the way it moves towards using visual practices that go beyond the usual limits of images and exhibition spaces.
	
\section{Brain functions}
	The region of the brain that is in charge of recibing, processing and interpretate the information that arrives trough our eyes is called visual region. As seen in Figure \ref{fig: Areas_of_the_brain}, the main structure for that is the visual cortex, placed on the back of the brain, in the occipital lobe. The light enters trough the eyes going trough the retina. In the retina, there are special cells (cones and rods) that transform the light into electrical signals. Those signals travel trough the optic nerve to a station called the thalamus (specifically to the lateral geniculate nucleus). Finally, the information arrive to the visual cortex.
	
	
	\begin{figure} [H]
		\centering
		\includegraphics[width=0.5\linewidth]{gfx/AreasOfTheBrain.jpg}
		\caption{Areas of the brain, by CogniFit Blog. \textit{Three main parts of the human brain}, 2015}
		\label{fig: Areas_of_the_brain}
	\end{figure}
	
	
	As it is explained in the first chapter of the book \textit{Neuroanatomy, Visual Cortex} \cite{Huff2023} the visual cortex is divided into five different areas (V1 to V5). As shown in Figure \ref{fig: Areas_of_the_visual_cortex}, those areas are classified according to their functional and structural characteristics. As information moves through the different areas, it becomes increasingly more specific. Neurons in the visual cortex are typically activated by stimuli within a specific receptive field. Each area contains neurons that are sensitive to different types of stimuli.
	
	\begin{figure} [H]
		\centering
		\includegraphics[width=0.5\linewidth]{gfx/AreasOfTheVisualCortex.png}
		\caption{Areas of the visual cortex, by  Leonard J. Press, O.D., FAAO, FCOVD. \textit{The Visual Centers of the Brain}, 2018}
		\label{fig: Areas_of_the_visual_cortex}
	\end{figure}
	
	
	\begin{itemize}
		\item \textbf{V1 - Primal visual cortex: \textsubscript} V1 is located around the calcarine sulcus. It consists of a laminar organization of six layers with a columnar architecture of marked ocular dominance and orientation.
		\\
		This area is the first cortical receptor of the retino-genicular cortical pathway. It receives information from the thalamus in layer 4 and sends projections to section V2 and other areas where all the information captured by V1 will be processed. 
		\\
		The main function of this area is to break down the image into its most basic concepts. It is responsible for detecting retinotopic positions, edge orientation, contrasts, spatial phases, and direction of movement. It also has a very precise retinotopy with very small and well-defined receptive fields, along with a fairly high spatial resolution.
		\\
		
		\item \textbf{V2 - Secondary Visual Area: \textsubscript} This region surrounds V1 and acts as an interface between V1 and other extrastriate areas. It continues to maintain rhinotopy, but its columns show a rather different organization from the previous region. It receives information from layers 2 and 3 of the V1 region and sends projections to V3, V4, and V5, as well as returning information to V1 for feedback. 
		\\
		This area is essential for the grouping of elements and the differentiation of figure and background. The main function of V2 is to integrate the output of V1. This area has larger receptive fields that allow it to reflect more complex properties than those detected previously, especially with the detection of textures, edges, color contrasts, and spatial frequency. Some types of cells in V2 participate in the detection of figures and boundaries. In addition, V2 participates in routing to the dorsal and ventral pathways.
		\\
		\item \textbf{V3 -  Visual Area 3: \textsubscript} Region V3 is located next to V2, on the dorsolateral surface of the occipital lobe. It receives signals from V1 and V2 and projects them to intermediate dorsal and ventral areas. 
		This region has even larger receptive fields than the first two and is responsible for deciphering and processing very complex spatial patterns that the other two areas are unable to process. It does not have a specific role or a fully designated function, but rather acts as an intermediate node for processing and transmitting complex information to the following areas. It is the area responsible for transmitting information to the following more specialized areas.
		\\
		
		\item \textbf{V4 - Colour specialized area: \textsubscript} The V4 region is located on the ventral and lateral surface of the occipital lobe. It is responsible for receiving inputs from V1, V2, and V3 and projects them to more ventral temporal areas.
		\\
		This region is responsible for color recognition, not only detecting the wavelength of the signal, but also registering color constancy under variations in lighting. By detecting and processing this information, it ensures that the brain reacts to colors according to the information detected in this region. It also has neurons that respond to curved shapes, corners, and combinations of lines and textures. It is responsible for visual attention. This part of the brain modulates strongly according to attention. It increases when attention is paid to a stimulus within the receptive field.
		\\
				
		\item \textbf{V5 - Movement Specialized area: \textsubscript} V5, also known as MT (middle temporal), is located in the lateral/dorsal temporal region and is responsible for receiving information from areas V1, V2, and V3 and sending it to regions of the dorsal parietal pathway.
		This region is responsible for detecting the direction and speed of movement, as its neurons are highly selective for this. It integrates local signals to encode global movement, an essential parameter for visual tracking and motion perception.
		\\
		
		\item \textbf{V8 - High level Color Processing: \textsubscript} This area, also called V4 plus, is located in the ventral region of the occipital lobe and was recently discovered through fMRI studies, when an area was observed that became highly active when color was perceived and analyzed.
		\\
		V8 is heavily involved in interpreting color, but in a much more complex way than V4. It specializes above all in achieving color constancy. It ensures that even if the intensity of the color changes, we can still identify the actual color being perceived. This ability requires complex calculations that depend not only on the information entering the retina, but also on spatial and contextual comparisons. It is a region that contributes to the identification of complex color patterns and boundaries defined by color differences.
	
	\end{itemize}
	
\section{Touch Designer}

	TouchDesigner is a real-time visual development platform developed by Derivative that is used extensively for the creation of interactive multimedia systems, data-driven visualizations, and generative art. The architectural framework of this system is predicated on a node-based procedural workflow, a methodology that empowers creators to construct intricate systems by establishing connections between functional units designated as operators \cite{derivativeND}. This modular approach is conducive to iterative design and facilitates the utilization of TouchDesigner by both artists and technically oriented users, aligning with the broader tradition of visual programming tools in new media \cite{harrington2016designing}.
	
	A significant advantage of TouchDesigner is its capacity for real-time rendering and data processing. This enables the manipulation of video, 3D geometry, audio, and sensor inputs with minimal delay. The software's capacity to perform such functions has led to its central role in the creation of large-scale audiovisual installations, projection mapping, interactivity in stage design, and immersive experiences \cite{bouchard2020interactive}. Its real-time nature situates the platform within contemporary practices of live media and performance technologies, where responsiveness and dynamic interaction are essential \cite{cox2019performance}.
	
	The TouchDesigner workflow is organized into specific categories of operator, namely TOPs, CHOPs, SOPs, DATs, and COMPs. Each operator category has been designed to process a particular data or execute a specific process. This layered structure enables creators to transition between surface-level interaction design and more profound computational logic \cite{kawaguchi2021touchdesigner}. The platform utilizes Python as its scripting environment, providing advanced control, automation, and logic-based behavior for interactive systems \cite{serrano2018python}. TouchDesigner's integration of visual programming with textual scripting positions it at the nexus of creative coding environments, such as Processing \cite{reas2014processing}, along with live coding paradigms that have been explored within performance contexts \cite{mclean2010livecoding}.
	
	
	
	 
	
	