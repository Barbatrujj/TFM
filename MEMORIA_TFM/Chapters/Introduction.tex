%************************************************
\chapter{Introduction}\label{ch:introduction}
%************************************************
	\section{Motivation and research problem}
		This project was motivated by a of academic interests and personal experiences. A strong curiosity for both audio and video has always been shared by us. Coming from backgrounds in audiovisual engineering and graphic design, we tend to approach creations with a critical and analytical eye. When it came time to choose a topic for our final master's project, we realized quickly that our interests were closely aligned. After engaging in numerous in-depth dialogues, we arrived at the decision to concentrate our research into two primary areas: the realm of visuals and music, along with the dynamic interplay between these two forms of expression. This intersection felt like the natural place where our skills, passions, and professional perspectives converged. 
		\\
		As we continued our research, we observed that in live shows, the visuals often had little or no meaningful connection to the music. Many visual productions are undeniably impressive. The work of Anyma is an example of this. However, these productions tend to function primarily as aesthetic elements. Rather than enhancing the auditory experience or accentuating the distinctive attributes of each composition, the visual elements often prove to be a source of distraction.
		
		We also discovered that a significant number of artists rely heavily on tutorials for creating their visuals. Tools like TouchDesigner, widely used in the industry, offer powerful capabilities for generating dynamic visual content. The platform offers numerous step-by-step tutorials aimed at beginners, guiding them through the creation of basic visual effects. However, we were surprised to find that even at major, internationally recognized festivals, some artists were using visuals that were essentially the same basic tutorial outputs, sometimes with only minimal modifications. This highlighted a gap between the potential of the technology and the creative depth actually being applied in many professional contexts.
		
	\section{Hypotesis and objectives}
		The issues surrounding the current use of visuals were recognized, and it was decided that this challenge would be made the foundation of our thesis. Since TouchDesigner has been a key tool throughout the Master's program and is widely regarded as a primary platform for creating visuals, we formulated the following research question:
		\textbf{Is it possible to develop a system within TouchDesigner that helps artists create visuals more easily?}
		
		\vspace{0.5cm}
		The rapid growth of artificial intelligence and the increasing accessibility of LLM-based tools, such as Microsoft Copilot and Cursor, are giving rise to new workflows that integrate AI directly into creative and professional environments. These advancements indicate that AI has the potential to play a significant role in supporting or even improving digital creation processes	Our concept is to incorporate artificial intelligence into TouchDesigner, empowering users to customize its training according to their unique requirements and artistic inclinations. This would enable artists to generate unique visuals more efficiently and reduce their dependence on generic tutorials, encouraging more personalized and innovative visual production.
		Based on that hypotesis, the next objectives were proposed to accomplish the project:
		\begin{itemize}
			\item \textbf{O\textsubscript{1} Research and training:} The first objective is to establish a solid research foundation for training LLM-based agents. The tool is intended to be customizable so that each artist can adapt it to their own creative workflow. This initial stage will involve selecting a specific artist as a reference case. The goal is to gather and structure the necessary information and then train the model to address that artist’s particular needs. This will not only demonstrate the approach's feasibility but also establish a framework for future users to customize the system to their specific artistic needs.
			
			\item \textbf{O\textsubscript{2}
			 Creation of the tool:} The second objective focuses on developing the tool within TouchDesigner. This stage involves building a network of AI-driven agents that are trained using the research compiled in the first objective. The goal is for this network to support a wide range of creative tasks, such as modifying colors and shapes, generating shaders, and producing audio-reactive visuals. Users should be able to accomplish these tasks by either providing an example visual or entering a prompt. In the end, this system tries to make the creative workflow more efficient and give artists the ability to create more personalized, expressive visuals more easily.
		\end{itemize}
	
	
	
%. If it were, the signal would begin to devirtualize and the resulting sound would no longer resemble the original.
%*****************************************
%*****************************************
%*****************************************
%*****************************************
%*****************************************




