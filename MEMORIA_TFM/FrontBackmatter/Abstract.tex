%*******************************************************
% Abstract
%*******************************************************
%\renewcommand{\abstractname}{Abstract}
\setlength{\parindent}{1cm}
\chapter*{Summary}
	The present project is situated within the domain of visuals in the entertainment industry. This visual resource has been found to be of significant value in a variety of works, productions and electronic music concerts. In recent years, the realm of visual creation has undergone significant advancements, largely attributable to the development of diverse software programs that have made it possible for any individual with access to a computer to produce remarkable visual content. Nevertheless, a considerable proportion of the visual content is characterised by its generic nature, simplicity, and absence of significance and intention. Frequently, the objective is merely to utilise the space or to enhance the visual appeal; however, there is an absence of a discernible intention underlying these actions. The present project is centred upon the utilisation of \ac{LLM} and Touch Designer models for the creation of a tool through which users can import information regarding their work, with the system subsequently generating visuals based on the imported data. This tool exhibits a modicum of autonomy, as it generates disparate visuals on each occasion from the same source of information. 
	
	The objective of the programme is to enable the user to input articles, references and information regarding their personal style. In this manner, the user is able to interact with the \ac{LLM} model within Touch Designer in a manner that is both intuitive and straightforward. Furthermore, the software can be complemented by numerous other compatible applications, including but not limited to Ableton, Unreal Engine, Photoshop, and Resolume Arena.
	


\newpage




\chapter*{Sumario}
		El presente proyecto se enmarca en el ámbito de los recursos visuales de la industria del entretenimiento. Estos recursos visuales han demostrado ser de gran valor en diversas obras, producciones y conciertos de música electrónica. En los últimos años, el ámbito de la creación visual ha experimentado avances significativos, en gran parte atribuibles al desarrollo de diversos programas de software que han permitido a cualquier persona con acceso a un ordenador producir contenidos visuales extraordinarios. Sin embargo, una parte considerable de los contenidos visuales se caracteriza por su naturaleza genérica, su simplicidad y su falta de significado e intención. A menudo, el objetivo es simplemente utilizar el espacio o mejorar el atractivo visual, pero no hay una intención discernible que subyazca a estas acciones. El presente proyecto se centra en la utilización de modelos \ac{LLM} y Touch Designer para la creación de una herramienta mediante la cual los usuarios pueden importar información sobre su trabajo, y el sistema genera posteriormente imágenes basadas en los datos importados. Esta herramienta muestra un mínimo de autonomía, ya que genera imágenes dispares en cada ocasión a partir de la misma fuente de información. 
		El objetivo del programa es permitir al usuario introducir artículos, referencias e información sobre su estilo personal. De esta manera, el usuario puede interactuar con el modelo \ac{LLM} dentro de Touch Designer de una manera intuitiva y sencilla. Además, el software se puede complementar con otras muchas aplicaciones compatibles, entre las que se incluyen, entre otras, Ableton, Unreal Engine, Photoshop y Resolume Arena.
	


\vfill